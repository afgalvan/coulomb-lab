\documentclass{article}
\usepackage[spanish]{babel}
\usepackage[utf8]{inputenc}
\usepackage{amsmath}
\usepackage{amsfonts}
\usepackage{amssymb}
\usepackage{mathtools}
\usepackage{minted}
\usepackage{gensymb}
\usepackage{xcolor}

\title{Formulas para el laboratorio de la ley de Coulomb}
\author{Andres Galvan}

\begin{document}
\maketitle

1. Formula de Coulomb
\begin{equation}
    \large F_{e} = k_{e}\frac{|q_{1}||q_{2}|}{r^{2}}
\end{equation}

2. Constante de Coulomb
\begin{equation}
\begin{split}
    \large k_{e} = 8.9876 \times 10^{9}N \cdot m^{2}/C^{2} \\
    \large k_{e} = \frac{1}{4\pi \varepsilon_{0} }
\end{split}
\end{equation}

3. Permitividad del vacío
\begin{equation}
    \large \varepsilon_{0} = 8.8542 \times 10^{-12}C^2/N\cdot m^2
\end{equation}

4. Relación entre F y r

A partir de la ecuación de la ley de Coulomb se puede estudiar la relación entre F 
y r. Considere la ecuación dada por:
\begin{equation}
    \large F_{e} = k_{e}\frac{|q_{1}||q_{2}|}{r^{2}}
\end{equation}

Esta ecuación se puede reescribir como: 
\begin{equation}
    \large F_{e} = (k_{e}|q_{1}||q_{2}|)\frac{1}{r^{2}}
    \large F_{e} = m\frac{1}{r^{2}}
\end{equation}

Donde: $m = k_{e}|q_{1}||q_{2}|$

De la expresión anterior se llega a extraer la relación de $F_{e}$ respecto a  $r$ como: 
\begin{equation}
    \large F_{e} = \propto \frac{1}{r^{2}}
\end{equation}

5. Porcentaje de error
\begin{equation}
    \large \% error = \frac{Valor\ aceptado - Valor\ experimental}{Valor\ aceptado} \times 100\%
\end{equation}
